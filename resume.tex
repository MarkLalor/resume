\documentclass[a4paper,11pt]{article}

%\usepackage[margin=0.05in]{geometry}

%A Few Useful Packages
\usepackage{marvosym}
\usepackage{fontspec} 					%for loading fonts
\usepackage{enumitem}
\usepackage{xunicode,xltxtra,url,parskip} 	%other packages for formatting
\RequirePackage{color,graphicx}
\usepackage[usenames,dvipsnames]{xcolor}
\usepackage[big]{layaureo} 				%better formatting of the A4 page
% an alternative to Layaureo can be ** \usepackage{fullpage} **
\usepackage{supertabular} 				%for Grades
\usepackage{titlesec}					%custom \section

%Setup hyperref package, and colours for links
\usepackage{hyperref}
\definecolor{linkcolour}{rgb}{0,0.1,0.2}
\hypersetup{colorlinks,breaklinks,urlcolor=linkcolour, linkcolor=linkcolour}

%FONTS
\defaultfontfeatures{Mapping=tex-text}
\setlist[itemize]{itemsep=-5pt, topsep=-8pt}
%\setmainfont[SmallCapsFont = Fontin SmallCaps]{
\setmainfont[
SmallCapsFont = Fontin-SmallCaps.otf,
BoldFont = Fontin-Bold.otf,
ItalicFont = Fontin-Italic.otf
]
{Fontin.otf}

\titleformat{\section}{\Large\scshape\raggedright}{}{0em}{}[\titlerule]
\titlespacing\section{0pt}{4pt plus 4pt minus 2pt}{4pt plus 2pt minus 2pt}

\geometry{top=0.4in,left=0.55in,bottom=0.4in,right=0.55in}

\begin{document}

\pagestyle{empty} % non-numbered pages

%--------------------TITLE-----------------------
\par{\centering{\Huge Mark Lalor}\par}
\begin{center}
\href{mailto:markwlalor@gmail.com}{markwlalor@gmail.com} • \href{https://linkedin.com/in/marklalor/}{linkedin.com/in/marklalor} • \href{http://github.com/marklalor}{github.com/marklalor}
\end{center}

%--------------------SECTIONS--------------------

% -------------- EDUCATION--- ----------
\section{Education}
\begin{tabular}{lp{15cm}}
\textbf{Case Western Reserve University} & \textsc{Sep} 2016 -- \textsc{Dec} 2019 \\
Bachelor of Science in Computer Science &  \textsc{GPA}: 3.74\\
\end{tabular} \\
\begin{tabular}{rp{16cm}}
\textsc{Classes}: & \footnotesize{Algorithms, Data Structures, Programming Language Concepts, Databases, Operating Systems, Artificial Intelligence,  Computer Security, Graph Theory, Computer Networks, Software Defined Networks}
\end{tabular}
% ---------- WORK EXPERIENCE ----------
\section{Work Experience}
\textsc{Software Engineer Intern} | \href{https://asana.com}{Asana} \\
\textsc{May 2019 – Aug 2019} | San Francisco, CA
\begin{itemize}
	\item Worked on frameworks and abstractions that product teams use for data loading.
	\item Reduced build times for product teams by restructuring large compilation targets related to the data model framework.
\end{itemize}
\vskip 2mm
\textsc{Research Assistant} | Case School of Engineering, \href{http://engr.case.edu/rabinovich_michael/}{Rabinovich} Lab \\
\textsc{Sep 2018 – Dec 2018} | Cleveland, OH
\begin{itemize}
	\item Conducted a measurement of the adoption of the EDNS Client Subnet (ECS) extension to DNS.
	\item Assessed the utility of the ECS extension for distributed platforms scanning.
\end{itemize}
\vskip 2mm
\textsc{Software Engineer Intern} | \href{https://www.mimsoftware.com}{MIM Software} \\
\textsc{May 2018 – Aug 2018} | Beachwood, OH
\begin{itemize}
	\item Implemented multi-factor authentication for an existing cloud software suite.
	\item Added options for time-based one-time passwords, SMS messages, and backup codes.
	\item Created new server-side services and added new HTTP endpoints for clients.
\end{itemize}
\vskip 2mm
\textsc{Teaching Assistant (Intro to Java)} | Case School of Engineering \\
 \textsc{Sep 2017 – Dec 2017} | Cleveland, OH
 \begin{itemize}
	 \item	Led two lab sections per week, assisting students completing the assignments.
	 \item Held a weekly office hour with a mini-lesson and project discussion.
	 \item Graded and provided feedback on lab assignments and projects.
 \end{itemize}
 \vskip 2mm
 \textsc{Software Engineer Intern} | \href{https://www.mimsoftware.com}{MIM Software} \\
 \textsc{May 2017 – Aug 2017} | Beachwood, OH
 \begin{itemize}
	\item Created new viewer/editor for DICOM (medical image) files that interfaces with main software.
	\item Maintained existing software components through version control and code review.
 \end{itemize}
\section{Leadership and Clubs}
\begin{itemize}[noitemsep]
	\item \emph{\href{http://hacsoc.org}{Hacker Society} Maintainer}: Organize weekly tech talks on campus. Reach out to current students as well as industry professionals and alumni to give talks.
	\item \emph{CWRU ACM}: Plan, manage, and volunteer for events that teach and support CS on campus. Provide guidance in regards to classes, the computer science department, and careers.
	\item \emph{Vietnamese Student Association}: Promote Vietnamese culture on campus and the Cleveland community by helping with events. Volunteered for a mentor/mentee program for new students.
\end{itemize}
\section{Skills}
\textsc{Programming Languages}: Java, Python, SQL, GraphQL, Typescript, Scala, C \\
\textsc{Software and Frameworks}: Git, Linux, Bazel, React, Redis
\section{Projects}
\begin{itemize}
    \item \emph{\href{https://github.com/marklalor/PLCInterpreter}{PLC Interpreter}}: Interpreter for a Java-like programming language with constructs such as objects, polymorphism, loops, recursion, and nested functions. Project in Programming Language Concepts.
    \item \emph{\href{https://github.com/marklalor/clanvas}{Clanvas}}: Command-line client for the Canvas learning management system. Uses the Canvas HTTP API to generate grade summaries, list assignments, and download course files.
    \item \emph{\href{https://github.com/ucfopen/canvasapi}{Canvasapi} and \href{https://github.com/python-cmd2/cmd2}{Cmd2}}: Open-source libraries used in Clanvas.
\end{itemize}

\end{document}
